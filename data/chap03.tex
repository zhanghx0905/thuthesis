% !TeX root = ../thuthesis-example.tex

\chapter{功能介绍}

\section{运行 CPP2PY}

CPP2PY 能以命令行工具和 Python API 两种方式运行,其主要输入参数如表 \ref{tab:3.1} 所示。这些输入参数可分为四类,指定输入输出文件信息的基本参数、传递给 Libclang 的输入解析参数、传递给构建脚本的编译参数、以及控制代码生成行为的其它参数。其中前三类参数的语义大多是显而易见的,第四类参数将在下文中介绍。

\begin{table}
  \centering
  \caption{CPP2PY 的主要输入参数}
  \begin{tabular}{lll}
    \toprule
     参数名                 &  参数解释                                  &  默认值   \\
    \midrule
     \lstinline$headers$                &  C++ 头文件路径                            &  -        \\
     \lstinline$sources$                &  C++ 实现文件路径                          &  空       \\
     \lstinline$modulename$             &  生成模块名                                &  -        \\
     \lstinline$setup_filename$         &  生成的构建脚本名                          &  setup.py \\
     \lstinline$target$                 &  生成文件路径                              &  当前路径 \\
     \lstinline$encoding$               &  文件编码                                  &  UTF8     \\
     \lstinline$incdirs$                &  Libclang 头文件包含路径                   &  空       \\
     \lstinline$libclang_flags$         &  传递给 Libclang 的额外参数,如 \lstinline$-DDEBUG$  &  空       \\
     \lstinline$libraries$              &  外部链接库                                &  空       \\
     \lstinline$library_dirs$           &  外部链接库路径                            &  空       \\
     \lstinline$compiler_flags$         &  编译选项                                  &  空       \\
     \lstinline$build$                  &  是否编译 Cython 代码                      &  \lstinline$True$     \\
     \lstinline$generate_stub$          &  是否生成 Python 存根文件 (.pyi)           &  \lstinline$True$     \\
     \lstinline$global_vars$            &  全局变量和宏所在对象名                    &  \lstinline$cvar$     \\
     \lstinline$registered_converters$  &  自定义的类型转换器                        &  空       \\
     \lstinline$renames_dict$           &  自定义的函数和变量重命名规则              &  空       \\
    \bottomrule
  \end{tabular}
  \label{tab:3.1}
\end{table}

通过命令行指令调用 CPP2PY,是打包简单 C/C++ 程序的不二之选。例如,下列指令将由头文件 modules.h 和代码文件 modules.cpp 组成的 C++ 项目打包成 Python 包,并命名为 cppproj。

\begin{framed}
\begin{lstlisting}[language=sh]
cpp2py modules.h --sources modules.cpp --modname cppproj
\end{lstlisting}
\end{framed}

如果有较为复杂的需求,如链接已有的动态共享库,则可以编写 Python 脚本,通过 API 调用 CPP2PY。下例指定了动态共享库的路径,如果不这样做的话,将在运行时因找不到共享库而引发导入异常。

\begin{framed}
\begin{lstlisting}[language=python]
from cpp2py import Config, make_cython_extention

config = Config(
    # ...
    library_dirs=[library_dir]
)
make_cython_extention(config)
\end{lstlisting}
\end{framed}

\section{函数}

当类型转换器模块提供了所有参数和返回值的处理方法时,C/C++ 函数被包装为 Python 函数。默认情况下,CPP2PY 支持的数据类型如表 \ref{tab:3.2} 所示。

\begin{table}
  \centering
  \begin{threeparttable}[c]
  \caption{数据在 C++ 与 Python 间的转换}
  \label{tab:3.2}
  \begin{tabular}{p{120pt}p{200pt}p{100pt}}
    \toprule
     Python =>                   &  C++                                                           &  => Python                 \\
    \midrule
     -                           &  \lstinline$void$                                                          &  \lstinline$None$                      \\
     基本数值类型   &  基本数值类型,包括布尔、整数、浮点、复数(\lstinline$std::complex$)  & 基本数值类型 \\
     \lstinline$array.array$, \lstinline$numpy.ndarray$  &  基本数值类型指针                                            &  被指向的数据              \\
     \lstinline$Iterable$                    &  基本数值类型定长数组                                        &  \lstinline$list$                      \\
     \lstinline$str$                         &  字符串(\lstinline$char *$, \lstinline$std::string$)                                 &  \lstinline$str$                      \\
     \lstinline$Iterable[str]$               &  字符数组(\lstinline$char **$)                                           &  被指向的字符串            \\
     \lstinline$Mapping$/\lstinline$Iterable$            &  STL 容器\tnote{①}  & \lstinline$set$/\lstinline$list$/\lstinline$dict$/\lstinline$tuple$    \\
     枚举类                      &  枚举                                                          &  枚举类                    \\
     类                          &  类、结构体、联合\tnote{②}                            &  类                        \\
     类                          &  类、结构体、联合的指针和二重指针                                    &  类                        \\
    \bottomrule
  \end{tabular}
  \begin{tablenotes}
    \item [①]  支持 \lstinline$vector$/\lstinline$set$/\lstinline$unordered_set$/\lstinline$map$/\lstinline$unordered_map$/\lstinline$pair$ 与基本数值类型、字符串的递归组合。

    \item [②]  作为返回值时要求类、结构体、联合具有拷贝构造函数。
  \end{tablenotes}
  \end{threeparttable}
\end{table}
% (
\subsection{可扩展的类型转换器}

CPP2PY 构建了一套具有可扩展性的类型转换器,用户可以通过继承抽象类 \lstinline{AbstractTypeConverter} 定义新的类型转换器,向程序提供新类型的转换方法,或重定义内置类型的转换行为。子类必须重写以下抽象方法:

\begin{itemize}
    \item \lstinline{matches}:根据参数类型和参数名判断本类型转换器是否适用于该参数,或根据返回值类型和函数名判断本类型转换器是否适用于该返回值。
    \item \lstinline{input_type_decl}:Cython 包装函数的输入参数类型;
    \item \lstinline{python_to_cpp}:规定如何将输入的 Python 对象转换为 C++ 参数;
    \item \lstinline{cpp_call_arg}:规定调用 C++ 接口时如何传递参数;
    \item \lstinline{return_output}:规定如何将 C++ 接口的返回值转换为 Python 对象并返回。
\end{itemize}


在五个抽象方法中,第二至第四个是用来处理函数参数的,而第五个用于处理返回值。除此之外,\lstinline{AbstractTypeConverter} 还定义了一些非抽象方法,必要时用户可以通过重写它们扩展类型转换器的功能。

\begin{itemize}
    \item \lstinline{add_imports}:规定需要额外导入的包,默认为空;
    \item \lstinline{pysign_type_decl}:规定在 Python 存根文件中如何标注类型,默认为 \lstinline{Any};
\end{itemize}

在 C 程序中,常常用 \lstinline{void *} 类型的变量充当泛型指针。有鉴于此,CPP2PY 内置了一个可选的 \lstinline{void *} 类型转换器 \lstinline{VoidPtrConverter},继承该类,指定 \lstinline{void *} 的实际指针类型,再将类型转换器子类以参数 \lstinline{registered_converters} 输入 Config 模块,就能让 CPP2PY 以处理基本数据类型指针的方式处理 \lstinline{void *}。

\begin{framed}
\begin{lstlisting}[language=python]
from cpp2py import VoidPtrConverter

class DataConverter(VoidPtrConverter):
    def real_type(self) -> str:
        return "double"
\end{lstlisting}
\end{framed}

\subsection{基本类型指针和数组}

Python 定义了一套包装底层连续内存的 C 层级标准,称为缓存协议。遵照缓存协议设计的数组对象之间可以快速转换,而不需要内存的复制,几乎知名 Python 库中的数组对象都实现了缓存协议,例如 Python 标准库 \lstinline{array},数值计算库 NumPy 中的多维数组。缓存协议是 Python 庞大科学计算生态的基石。

基于缓存协议,Cython 提供了一种特殊的数据结构,带类型的内存视图(typed memoryview),能够与 C 数组、Python 数组或 NumPy 数组相互转换。CPP2PY 使用这种数据结构来向 C 接口中传递基本数据类型的指针。

举例而言,假如有如下函数定义,

\begin{framed}
\begin{lstlisting}[language=c++]
double norm2(double[] vec, unsigned size);
void increment(int* i);
\end{lstlisting}
\end{framed}
用户将能在 Python 端以标准库数组或 NumPy 数组的形式向底层接口传递数组和指针参数。
\begin{framed}
\begin{lstlisting}[language=python]
>>> from array import array
>>> arr = array('d', [4., 3.])
>>> norm2(arr, len(arr))
5.0
>>> import numpy as np
>>> arr = np.array([1], dtype=np.int32)
>>> increment(arr); arr
array([2], dtype=int32)
\end{lstlisting}
\end{framed}

如果输入的数组元素类型与底层接口不匹配,将引发异常。

\subsection{引用和常量}

在 Python 中,没有与右值引用和常量相对应的概念,因此函数参数和返回值的右值引用标识符 \lstinline{&&} 和常量标识符 \lstinline{const} 将被忽略。

对于左值引用参数,只有当它指向一个类时,对引用的修改才会反映到输入参数上,因为此时 C++ 接口的输入参数直接来自于代理类,而在其他情况下,输入的 Python 对象会被先转换成 C++ 变量,对 C++ 变量的修改不会反作用到原 Python 对象上。

CPP2PY 不特殊处理引用类型返回值。当然,用户可以通过自定义类型转换器来修改 CPP2PY 的行为。

\subsection{默认参数}

CPP2PY 支持默认参数语法,受支持的默认值类型包括基本数据类型,即整数、浮点数、布尔值,和字符串。

\begin{framed}
\begin{lstlisting}[language=c++]
double mult(double i = 5, int j = 6ull) { return i * j; }
\end{lstlisting}
\end{framed}对应的 Python 接口,\begin{framed}
\begin{lstlisting}[language=python]
>>> mult()
30.0
>>> mult(j = 4, i = 1.2)
4.8
\end{lstlisting}
\end{framed}

如果一个参数的默认值不被 CPP2PY 所支持,那它左侧参数的默认值也将被忽略。举个例子,

\begin{framed}
\begin{lstlisting}[language=c++]
int divide(int a = 1, int* b = nullptr);
\end{lstlisting}
\end{framed}对应的 Python 接口中,参数 \lstinline{a} 的默认值也将被忽略。

\subsection{函数重载与重命名字典}

函数重载是 C++ 的一大特色,但 Cython 只支持声明 C++ 重载函数,不支持重载 Python 函数。默认情况下,CPP2PY 只会选取一个受类型转换器支持的函数、方法和构造函数进行封装,而忽略它的重载版本。

为了解决这个问题,CPP2PY 引入了重命名字典的概念,用户可以以 API 参数的形式对函数进行重命名,比如,如果你有这样两个函数,

\begin{framed}
\begin{lstlisting}[language=c++]
void foo(int c) { cout << c << endl; };
void foo(char *c) { cout << c << endl; };
\end{lstlisting}
\end{framed}

默认情况下,CPP2PY 只会包装两个重载函数中的一个,但是通过重命名字典,就可以将它们区分开来。

\begin{framed}
\begin{lstlisting}[language=python]
renames_dict = {
    ("foo", "void(int)"): "foo_int",
    ("foo", "void(char*)"): "foo_str",
}
\end{lstlisting}
\end{framed}

重命名字典的键为一个元组,元组的第一个元素为函数名,第二个元素为函数类型,对应的值为函数的新名称;元组也可以只由一个元素构成,在这种情况下,CPP2PY 会重命名所有名称相匹配的函数。

\begin{framed}
\begin{lstlisting}[language=python]
>>> foo_int(3)
3
>>> foo_str("3")
3
\end{lstlisting}
\end{framed}

重命名字典也可作用于类的方法,但不能作用于类的构造函数。

\begin{framed}
\begin{lstlisting}[language=c++]
class A {
    void foo(int c);
    void foo(char *c);
}

/* 
renames_dict = {
    ("A::foo", "void(int)"): "foo_int",
    ("A::foo", "void(char*)"): "foo_str",
}
*/
\end{lstlisting}
\end{framed}

\section{全局变量与宏}

为了提供对 C/C++ 全局变量的访问,CPP2PY 创建了一个特殊的对象 \lstinline{cvar}。例如,如下 C++ 全局变量,

\begin{framed}
\begin{lstlisting}[language=c++]
const int My_variable = 5;
double density = 0.1;
\end{lstlisting}
\end{framed}在 Python 中的接口如下,\begin{framed}
\begin{lstlisting}[language=python]
>>> cvar.My_variable
5
>>> cvar.density
0.1
>>> cvar.density = 0.9
\end{lstlisting}
\end{framed}

给常量赋值,或赋的值类型与 C++ 变量类型不匹配,将引发异常。

\begin{framed}
\begin{lstlisting}[language=python]
>>> cvar.My_variable = 0
AttributeError: attribute 'My_variable' of 
    'globals._cvar' objects is not writable
>>> cvar.density = "Hello"
TypeError: must be real number, not str
\end{lstlisting}
\end{framed}

CPP2PY 能解析定义了基本数值类型和字符串字面量的宏,并将它们视作不带作用域的常量。

\begin{framed}
\begin{lstlisting}[language=c++]
#define GET_LUCKY "We're up all night to get lucky"
\end{lstlisting}
\end{framed}

如果我们不想使用 \lstinline{cvar} 这个名字来访问 \lstinline{GET_LUCKY},可以通过命令行或 API 参数指定一个新的名称。

\begin{framed}
\begin{lstlisting}[language=sh]
$ cpp2py [filename] --globals myvar

>>> myvar.GET_LUCKY
"We're up all night to get lucky"
\end{lstlisting}
\end{framed}

即便模块中没有任何需要包装的客体,CPP2PY 也会创建 \lstinline{cvar} 对象。

\section{枚举}

定义一组具有相关语义的常量时,枚举非常有用。典型的例子包括星期几(周日到周六)和课程成绩(“A”到“D”和“F”)。在 C++ 中,枚举实质上是在一定作用域下的整数常量。而在 Python 中,枚举类通过继承标准库中的枚举基类实现。

CPP2PY 将 C/C++ 中的枚举映射为 Python 枚举类,受支持的语法包括带作用域枚举,嵌套枚举等,唯一不受支持的语法是匿名枚举。对于如下 C++ 代码,

\begin{framed}
\begin{lstlisting}[language=c++]
enum class Suit { Diamonds,
    Hearts,
    Clubs,
    Spades };
typedef enum { Hit,
    Miss } Result;

Result guess_card(Suit suit)
{
    if (suit == Suit::Clubs) {
        return Hit;
    }
    return Miss;
}
\end{lstlisting}
\end{framed}

其对应的 Python 接口如下,

\begin{framed}
\begin{lstlisting}[language=python]
>>> Result.Miss
<Result.Miss: 1>
>>> guess_card(cppenum.Suit.Clubs)
<Result.Hit: 0>
\end{lstlisting}
\end{framed}

\section{类、结构体与联合}

C++ 中的类、结构体和联合被包装在 Python 代理类中,一个代理类对象包括两部分,分别是包含着指向底层 C++ 对象指针的 Python 对象,以及底层的 C++ 对象。以下所有对类的讨论同样适用于结构体和联合。

CPP2PY 将 C++ 类中的方法、数据成员映射为 Python 代理类中的方法和属性,常量数据成员将被映射为不可变属性。

静态方法的语法在 Python 中得到了保留,但静态数据成员将被包装为全局变量。例如,对于下列声明,

\begin{framed}
\begin{lstlisting}[language=c++]
class A {
public:
    static int count;
    static int get_count() { return count; }
};
\end{lstlisting}
\end{framed}

在 Python 端以访问全局变量的方式访问类的静态数据成员,

\begin{framed}
\begin{lstlisting}[language=python]
>>> cvar.count = 5
>>> A.get_count()
5
\end{lstlisting}
\end{framed}

CPP2PY 能够解析嵌套类,但是不支持匿名结构体和匿名联合。

\subsection{类的构造}

CPP2PY 会忽略掉 C++ 类的拷贝构造函数和移动构造函数。此外,构造函数的重载不被支持,在受类型转换器支持的所有构造函数中,只有一个会被包装为代理类的构造函数。

根据 C++ 语法规则,如果一个类没有显式声明任何构造函数,CPP2PY 将自动为其添加一个默认构造函数。

有些类确实没有可用的构造函数,比如具有纯虚方法的抽象类,以及没有公开构造函数的类。在这种情况下,CPP2PY 将在 Python 端生成一个特殊的构造函数,在 Python 对象初始化时抛出异常。

\begin{framed}
\begin{lstlisting}[language=c++]
class AbstractClass {
protected:
    AbstractClass() { }

public:
    virtual ~AbstractClass() { }
    virtual double square() = 0;
};
\end{lstlisting}
\end{framed}

尝试在 Python 端构造抽象类,会引发异常。

\begin{framed}
\begin{lstlisting}[language=python]
>>> AbstractClass()
TypeError: Can't instantiate class AbstractClass 
    for no available constructors.
\end{lstlisting}
\end{framed}

\subsection{继承}

CPP2PY 处理继承的方式是将超类的公有方法和数据成员拷贝到派生类中。这种做法保证了生成代码的简洁,也能处理多重继承。但代价是类的继承关系将不能得到保持。以下列代码为例,


\begin{framed}
\begin{lstlisting}[language=c++]
class Shape {
public:
    virtual ~Shape() {};
    virtual double area() = 0;
    virtual double perimeter() = 0;
};

class Circle : public Shape {
    int radius;

public:
    Circle(double radius) : radius(radius) {};
    double area();
    double perimeter();
};

class Square : public Shape {
    double size;

public:
    Square(double size) : size(size) {};
    double area();
    double perimeter();
};

\end{lstlisting}
\end{framed}

对应 Python 代理类的继承关系将不能成立。

\begin{framed}
\begin{lstlisting}[language=python]
>>> issubclass(Circle, Shape), issubclass(Square, Shape)
(False, False)
\end{lstlisting}
\end{framed}

在 C++ 中,常常用基类的指针包裹派生类对象,这种多态性在函数参数中仍能得到保持,这个特性是利用 Cython 的混合类型(Fused Type)语法实现的。

\begin{framed}
\begin{lstlisting}[language=python]
# void print_shape(Shape* shape);
>>> print_shape(Circle(radius=2))
area = 12.5664
perimeter = 12.5664
>>> print_shape(Square(size=2))
area = 4
perimeter = 8
\end{lstlisting}
\end{framed}

但如果类似的情况出现在函数返回值中,CPP2PY 就无法处理,因为它不能确定对象的实际类型,而 Python 又没有将基类对象转换成派生类对象的语法。在这种情况下,用户需要根据需求修改 CPP2PY 生成的 Cython 代码。

对代理类来说,只有公有继承是有意义的,因为私有继承和保护继承不能带来新的公有成员。

CPP2PY 不支持以 \lstinline{using} 声明将基类非公有成员公有化的语法。

\begin{framed}
\begin{lstlisting}[language=c++]
class Derive : public Base {
public:
    using Base::foo;	// foo is private in Base
}
\end{lstlisting}
\end{framed}

\subsection{运算符重载}

\begin{table}
  \centering
  \caption{重载运算符的转换}
  \begin{tabular}{p{80pt}p{100pt}p{200pt}}
    \toprule
     类型          &  C++                    &  Python                                                       \\
    \midrule
     数值运算      &  \lstinline$+$,\lstinline$-$,\lstinline$*$,\lstinline$/$,\lstinline$%$          &  \lstinline$__add__$、\lstinline$__sub__$、\lstinline$__mul__$、\lstinline$__truediv__$、\lstinline$__mod__$    \\
     就地数值运算  &  \lstinline$+=$,\lstinline$-=$,\lstinline$*=$,\lstinline$/=$,\lstinline$%=$     &  \lstinline$__iadd__$、\lstinline$__isub__$、\lstinline$__imul__$、\lstinline$__itruediv__$、\lstinline$__imod__$ \\
     位运算        &  \lstinline$&$,\lstinline$|$ ,\lstinline$~$,\lstinline$^$,\lstinline$<<$,\lstinline$>>$    &  \lstinline$__and__$、\lstinline$__or__$、\lstinline$__invert__$、\lstinline$__xor__$、\lstinline$__lshift__$、\lstinline$__rshift__$ \\
     就地位运算    &  \lstinline$&=$,\lstinline$|=$,\lstinline$^=$,\lstinline$<<=$,\lstinline$>>=$  &  \lstinline$__iand__$、\lstinline$__ior__$、\lstinline$__ixor__$、\lstinline$__ilshift__$、\lstinline$__irshift__$ \\
     比较运算      & \lstinline$<$,\lstinline$>$,\lstinline$<=$,\lstinline$>=$,\lstinline$==$,\lstinline$!=$   &  \lstinline$__lt__$、\lstinline$__gt__$、\lstinline$__le__$、\lstinline$__ge__$、\lstinline$__eq__$、\lstinline$__ne__$   \\
     函数调用      &  \lstinline$()$                     &  \lstinline$__call__$                                                   \\
     其他运算符    &                         &  通过重命名字典指定                                           \\
    \bottomrule
  \end{tabular}
  \label{tab:3.3}
\end{table}

C++ 类的重载运算符将被尽可能的映射到 Python 中。Python 类的每个运算符都对应一个约定的魔法方法,魔法方法的特点是以双下划线开头和结尾。

然而,由于两种语言不同的语法设计,CPP2PY 不能自动处理某些运算符的包装。首先,Python 不允许重载逻辑运算符 \lstinline{and(&&)}、\lstinline{or(||)} 、\lstinline{not (!)} 和赋值运算符,这类算符将会被忽略;其次,C++ 下标运算符对应 Python 中的两个魔法方法, \lstinline{__getitem__} 和 \lstinline{__setitem__},用户应该通过重命名字典进行指定。另外,就地运算符的语义差别也值得注意,在 C++ 中,\lstinline{a += b}  表示 \lstinline{a.operator+=(b)},而 Python 在执行就地操作时 还将执行一步额外的赋值操作,即 \lstinline{a = a.operator+=(b)}。

重载运算符在 C++ 和 Python 间的转换关系如表 \ref{tab:3.3} 所示。CPP2PY 只能处理以类方法形式实现的运算符重载,以友元函数实现的运算符重载将会被忽略。



\subsection{对象的所有权管理}

C/C++ 和 Python 以两种截然不同的方式管理对象的生命周期。 在 C++ 中,每一块的堆内存都需要被手动释放;而在 Python 中,对象的生命周期完全由垃圾回收器掌控,当一个对象的引用计数为 0 时, 它占用的内存将被回收,具体的回收时机由垃圾回收器在运行时确定。

使用代理类时需要注意的一个主要问题是包装对象的内存管理。考虑下面的 C++ 代码,

\begin{framed}
\begin{lstlisting}[language=c++]
class Foo {
	...
};

class Spam {
public:
    Foo *value;
    ...
};
\end{lstlisting}
\end{framed}

假如在 Python 中这样使用这些类,

\begin{framed}
\begin{lstlisting}[language=python]
f = Foo()
s = Spam() 
s.value = f
g = s.value
g = 4
del f
\end{lstlisting}
\end{framed}

现在,考虑由此产生的内存管理问题。创建代理类对象时,既创建了代理类实例,又创建了底层的 C++ 实例,f 和 s 都是如此。赋值语句 \lstinline{s.value = f} 将 f 内部的指针存储到 s 底层的 C++ 对象中。\lstinline{g = s.value} 创建了第二个 \lstinline{Foo} 代理类对象,它也包裹了 f 的指针。现在,两个代理类对象和一个 C++ 对象共享了同一个指针。对 g 重新赋值,旧的 g 值引用计数为 0,这将导致一个代理类对象被回收;而紧接着 \lstinline{del f} 又销毁了一个代理类对象。 此时 \lstinline{s.value} 仍然保存着 f 内部的指针,它是否还有效呢?

为了解决这个问题,CPP2PY 为代理类创建了 \lstinline{owner} 属性来控制其对底层 C++ 对象的所有权。当代理类对象被析构时,只有 \lstinline{owner} 属性为真时,底层的 C++ 对象才会被销毁。当一个新的代理类对象被创建时,\lstinline{owner} 被设置为真。当对象的指针被返回时,它被一个代理类包装,如果该指针来自一个 \lstinline{getter} 函数,那代理类对象将不具有所有权。

仍以刚才的代码为例,只要在析构 f 之前将其 \lstinline{owner} 属性置为否,则 f 析构后 \lstinline{s.value} 就仍是合法的。

\begin{framed}
\begin{lstlisting}[language=python]
>>> f = Foo(); s = Spam(); s.value = f; g = s.value
>>> g.owner
False
>>> f.owner
True
\end{lstlisting}
\end{framed}

\section{其它语法特性}

\subsection{命名空间}

命名空间是 C++ 用来解决名称冲突问题的语法。一个标识符可在多个命名空间中定义,不同的命名空间的同名标识符互不相干。

CPP2PY 能解析 C++ 的命名空间。但是命名空间既不会出现在模块中,也不会导致模块被分解成子模块。如果你的程序存在多个命名空间,要小心名称冲突问题。函数和方法的名称冲突可以用重命名字典解决,而对于其它标识符的名称冲突问题,只能通过修改 C++ 源代码来手动解决。

\subsection{类型别名}

CPP2PY 支持解析以 \lstinline{typedef}、宏和 \lstinline{using} 定义的类型别名,但不支持模板别名,以下列接口为例,

\begin{framed}
\begin{lstlisting}[language=c++]
typedef double mytype;
#define MYTYPE mytype
using mytype2 = const MYTYPE;

mytype2 fun(mytype d) { return d + 1.0; }
\end{lstlisting}
\end{framed}

对应到 Python 中,

\begin{framed}
\begin{lstlisting}[language=python]
>> fun(2)
3.0
\end{lstlisting}
\end{framed}


但需要注意的是,CPP2PY 不会将类型别名映射到 Python 端。

\subsection{异常}

C++ 和 Python 都支持异常的抛出和捕获,在跨语言项目中,统一的错误处理风格有助于编写更简洁有效的程序。

得益于 Cython,CPP2PY 支持将 C++ 异常传递到 Python 端,并将一部分 C++ 的内置异常类映射为 Python 的内置异常类,映射关系如表 \ref{tab:3.4} 所示。

\begin{table}
  \centering
  \caption{异常在 C++ 和 Python 间的转换}
  \begin{tabular}{ll}
    \toprule
     C++                                           &  Python            \\
    \midrule
     \lstinline$std::bad_alloc$                              &  \lstinline$MemoryError$     \\
     \lstinline$std::bad_cast$、\lstinline$std::bad_typeid$            &  \lstinline$TypeError$       \\
     \lstinline$std::domain_error$、\lstinline$std::invalid_argument$  &  \lstinline$ValueError$      \\
     \lstinline$std::ios_base::failure$                      &  \lstinline$IOError$         \\
     \lstinline$std::out_of_range$                           &  \lstinline$IndexError$      \\
     \lstinline$std::overflow_error$                         &  \lstinline$OverflowError$   \\
     \lstinline$std::range_error$、\lstinline$std::underflow_error$    &  \lstinline$ArithmeticError$ \\
     其他异常                                      &  \lstinline$RuntimeError$    \\
    \bottomrule
  \end{tabular}
  \label{tab:3.4}
\end{table}


\subsection{兼容 C}

CPP2PY 默认生成 C++ 风格接口,比如使用 \lstinline{new} 而不是 \lstinline{malloc} 来构造结构体实例,并将调用 C++ 编译器编译目标代码。而 C++ 编译器为了支持重载、命名空间等语法特性,会对函数和方法做名称修饰(name mangling),如果你的项目是以 C 语言编写的,这将在模块导入时引发链接错误。

有一种简单的方式来让 C 项目获得 CPP2PY 的支持,即在需要封装的接口前添加 \lstinline{extern "C"} 链接声明。